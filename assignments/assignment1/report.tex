\documentclass[letterpaper]{article}
\synctex=1
\usepackage{graphicx}
\graphicspath{ {images/} }

\usepackage{lipsum}
\usepackage{float}

% \usepackage[
%     style=ieee,
%     backend=biber
%     ]{biblatex}
% \addbibresource{references.bib}

\usepackage{hyperref}

\usepackage{amssymb}

\usepackage{siunitx}

\usepackage{multirow}
% for merging table cells I think

\usepackage{tabularx}
\renewcommand\tabularxcolumn[1]{m{#1}}% for vertical centering text in X column
% allows for linewrap within cells
\newcolumntype{Y}{>{\centering\arraybackslash}X}

\usepackage{todonotes}
\usepackage{pdfpages}

\usepackage{fancyhdr} %header
\fancyhf{}
\fancyhead[R]{Arun Woosaree XXXXXXX}
\renewcommand\headrulewidth{0pt}
\fancyfoot[C]{\thepage}
\renewcommand\footrulewidth{0pt}
\pagestyle{fancy}

\usepackage[pdf]{graphviz}
\usepackage{adjustbox}

\usepackage{amsmath}

% make subsection use letters
\renewcommand{\thesubsection}{\alph{subsection})}

\usepackage{minted}

% \usepackage{amsthm}
\title{ECE 421 \\
Assignment 1}
  \author{Arun Woosaree\\
  XXXXXXX}
%actual document
\begin{document}

\maketitle %insert titlepage here

\section{}
Functional programming is a declarative paradigm.
Computation is treated as the evaluation of mathematical functions.
Expressions and declarations are used as opposed to statements.
Given a set of inputs, the output will always be the same.
Data is immutable, and as a result,
functions have no 'side-effects'.

\section{}
\todo{Explain what it does}

Right at the beginning, we see that the program produces a side-effect, which
is using IO to output some data. This contradicts the idea of a 'purely'
functional language. However, if a program never has side effects, it would be
pretty boring. \dots \todo{say something more about side effect}

We see that the program is written in a declarative style, even though Haskell
is generally considered a purely functional language\dots


\section{}
Immutability is nice, because it makes things predictable\dots

\section{}
\begin{minted}[]{c}
void add(&int x, int y, int z){
  *x = y + z;
}
\end{minted}
%\begin{adjustbox}{width=\textwidth}
%\lstinputlisting[language=Python]{stationaryprob.py}
%\end{adjustbox}
%\vspace{2cm}



\end{document}
