\documentclass[letterpaper]{article}
\synctex=1
\usepackage{graphicx}
\graphicspath{ {images/} }

\usepackage{lipsum}
\usepackage{float}

% \usepackage[
%     style=ieee,
%     backend=biber
%     ]{biblatex}
% \addbibresource{references.bib}

\usepackage{hyperref}

\usepackage{amssymb}

\usepackage{siunitx}

\usepackage{multirow}
% for merging table cells I think

\usepackage{tabularx}
\renewcommand\tabularxcolumn[1]{m{#1}}% for vertical centering text in X column
% allows for linewrap within cells
\newcolumntype{Y}{>{\centering\arraybackslash}X}

\usepackage{todonotes}
\usepackage{pdfpages}

\usepackage{fancyhdr} %header
\fancyhf{}
\fancyhead[R]{Arun Woosaree XXXXXXX}
\renewcommand\headrulewidth{0pt}
\fancyfoot[C]{\thepage}
\renewcommand\footrulewidth{0pt}
\pagestyle{fancy}

\usepackage[pdf]{graphviz}
\usepackage{adjustbox}

\usepackage{amsmath}

% make subsection use letters
\renewcommand{\thesubsection}{\alph{subsection})}

\usepackage{minted}

% \usepackage{amsthm}
\title{ECE 421 \\
Assignment 1}
\author{Arun Woosaree\\
XXXXXXX
}

%actual document
\begin{document}

\maketitle %insert titlepage here

\section{}
Functional programming is a declarative paradigm.
Computation is treated as the evaluation of mathematical functions.
Expressions and declarations are used as opposed to statements.
Given a set of inputs, the output will always be the same.
Data is immutable, and as a result,
functions have no 'side-effects'.

\section{}
The function is summing the numbers from 0 to 100.
Right at the beginning, we see that the program produces a side-effect, which
is using IO to output some data. This contradicts the idea of a 'purely'
functional language. However, if a program never has side effects, it would be
pretty boring.
Also, 
we see that the program is written in an imparative style, even though Haskell
is generally considered a purely functional language where code is usually
written in a declarative style. 
That is, it looks like a C programmer wrote some code here.
The looping part of the program is recursive,
and the value i is updated using a monad. The program still is functional,
in the sense that the function does the same thing every time, and the dirty
work of updating variables is achieved with the monad. Also, it does not mutate
anything outside of the function.


\section{}
Immutability is nice, because it makes things predictable.
For example, in concurrent applications, shared data is a lot more predictable,
since nothing can mutate the data. 
Immutability also helps with producing more readable code, because values don't
change. Also, when you call a function, you know that it has not mutated
anything. This is what makes things more predictable.

\section{}
\subsection{}
\begin{minted}[]{c}
  int x = y + z;
\end{minted}

\subsection{}
\begin{minted}[]{haskell}
  let x = y + z
\end{minted}

\subsection{}
All languages must allow for mutability at some level, otherwise they would
never produce useful output. When code is compiled, it generates mutable machine
code since that's what computers (currently) understand.
\section{}

\subsection{}
\begin{itemize}
  \item sqrtx takes an input, and returns the squared value of the input
  \item imparativefun returns the sum of squares of its input, which is a list of numbers
  \item functionalfun does the same thing, but does so in a declarative manner instead of an imparative way
\end{itemize}

\subsection{}
\begin{itemize}
  \item there seems to be no difference in reliability between the two
    implementations
  \item there may be a slight difference in efficiency between the two
    implementations, but it is probably negligible
  \item I would argue that the functionalfun is more maintainable as it's easier to
understand what the function is doing. Even though the functions do the
same thing, functionalfun has less ``clutter'' around it, which is what makes it
easier to understand.
\item there will be no difference in portability between the two
  implementations.
\end{itemize}

\subsection{}
\begin{minted}[]{fsharp}
  let fourthpower x = x |> sqrt |> sqrt
\end{minted}

\section{}
\begin{itemize}
  \item changing the file system is a side effect, so it is not a pure function
  \item inserting a record cannot be a pure function, because it results in the
    database mutating
  \item making an http call cannot be a pure function because you cannot
    guarantee the same thing will be returned on every function call. The status
    message can change, the server could go down, etc.
  \item printing to the screen requires a side-effect, namely I/O, so it is not
    a pure function
  \item querying the DOM can be a pure function, provided the DOM is an input
    parameter to the function, and the output of the function depends only on
    this input, and if the function does not mutate anything
  \item the system state will be different (for example, the time will be
    different) on every call to the function, so it
    will return something different each time, so it cannot be a pure function
  \item pure functions return the same output every time for a given set of
    inputs. Any randomness violates this, so Math.random() is not a pure
    function
\end{itemize}

\section{}
\inputminted[]{rust}{7.rs}

\section{}
\inputminted[]{rust}{8.rs}

\section{}
The Scheme function returns a sum of the number of leaf nodes in a nested list
that are not \#f. That is, no matter how nested the elements in the lists are,
it returns a count of all the elements in a list, and its nested lists not 
including \#f.
%It does so recursively.
%If the empty list is passed in, the function x will return 0.
%Next, if the input is a pair and car isn't a list:\\
%   if car is \#f, then recurse on cdr\\
%   else recurse on cdr and add 1\\
%
%   lastly, recurse on car and cdr, and add the results.







\section{}
\subsection{}
The GHC compiler produces no error when presented this code. This is because
Haskell supports things like partial function applications and incomplete pattern matching.
The program will compile with no errors by default, but the flag
\texttt{-fwarn-incomplete-patterns}
can be used to show a warning at compile time.

\subsection{}
The Rust compiler will scream at you for this. It will catch that the pattern
Blue defined by Colour is not covered in the pattern matching statement and tell
you to fix it. Till then, it will refuse to compile the program.
\end{document}
